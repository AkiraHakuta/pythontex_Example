%TrigonometricTable.tex    Author: Akira Hakuta, Date: 2016/12/04
%pdflatex.exe -synctex=1 -interaction=nonstopmode TrigonometricTable.tex
%pythontex.exe  TrigonometricTable.tex
%pdflatex.exe -synctex=1 -interaction=nonstopmode TrigonometricTable.tex

\documentclass[a4paper, twocolumn, 11pt]{article}
%\documentclass[a4paper, twocolumn, 11pt, jadriver=standard]{bxjsarticle}

\setlength{\textwidth}{195mm}
\setlength{\textheight}{265mm} 
\setlength{\voffset}{0mm}
\setlength{\topmargin}{-15mm}
\setlength{\headheight}{8mm}
\setlength{\headsep}{0mm}
\setlength{\columnsep}{5mm}
\setlength{\evensidemargin}{0mm}
\setlength{\oddsidemargin}{-17mm}
\usepackage{setspace}

\usepackage{fancyhdr}
\pagestyle{fancy}
\chead{}
\renewcommand{\headrulewidth}{0.0pt}
\cfoot{}
\setlength{\columnseprule}{0.0pt}

\usepackage{pythontex}

\begin{document} 
\lhead{\bf\Large  ~~~Trigonometric~Table}
   
\begin{pycode}
import math

step = 0.5 #0.1 0.2 0.5 1.0
start, end = 0, 90
line_num = 45

step10=int(step*10)
start10, end10 = start*10, end*10

def one_page_print(start1, step1):
    x10 = start1
    print(r"\begin{table}")
    print(r"\setstretch{1.1}")
    print(r"\begin{center}")
    print(r"\begin{tabular}{|c|c|c|c|}")
    print(r"\hline")
    print(r"$~~~x~~~$ & $~~~~\sin x~~~~$& $~~~~\cos x~~~~$ & $~~~~\tan x~~~~$  \\ \hline")
    for line in range(line_num + 1):
        rad = x10*math.pi/180 /10
        if x10== end10:
            print(r"{:4.1f}$^\circ$ & {:8.6f}  & {:8.6f} & undefined\\".format(x10/10, math.sin(rad), math.cos(rad)))
        else:
            if math.tan(rad) <100:
                print(r"{:4.1f}$^\circ$ & {:8.6f}  & {:8.6f} & {:8.6f}\\".format(x10/10, math.sin(rad), math.cos(rad), math.tan(rad)))                
            else:
                print(r"{:4.1f}$^\circ$ & {:8.6f}  & {:8.6f} & {:8.3e}\\".format(x10/10, math.sin(rad), math.cos(rad), math.tan(rad)))                
        if line%5 == 0:
            print(r"\hline")
        x10 += step1
        if x10 > end10:
           print(r"\end{tabular}") 
           print(r"\end{center}")
           print(r"\end{table}")
           return 
    print(r"\end{tabular}") 
    print(r"\end{center}")
    print(r"\end{table}") 


x10 = start10
while True:
    if x10 == end10:
        break
    one_page_print(x10, step10)    
    x10 += step10*line_num       
 
\end{pycode} 
                
\end{document}
